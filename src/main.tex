\documentclass[11p]{article}
% Packages
\usepackage{amsmath}
\usepackage{graphicx}
\usepackage{fancyheadings}
\usepackage[swedish]{babel}
\usepackage[
    backend=biber,
    style=authoryear-ibid,
    sorting=ynt
]{biblatex}
\usepackage[utf8]{inputenc}
\usepackage[T1]{fontenc}
%Källor
\addbibresource{references.bib}
\graphicspath{ {./images/} }

% Lite variabler
\def\email{malte.lindkvist@ga.ntig.se}
\def\foottitle{PMmall}
\def\name{Malte Oskar August Lindkvist}

\title{Bakgrunds PM \\ \small Gymnasiearbete}
\author{\name}
\date{\today}

\begin{document}

% fixar sidfot
\lfoot{\footnotesize{\name \\ \email}}
\rfoot{\footnotesize{\today}}
\lhead{\sc\footnotesize\foottitle}
\rhead{\nouppercase{\sc\footnotesize\leftmark}}
\pagestyle{fancy}
\renewcommand{\headrulewidth}{0.2pt}
\renewcommand{\footrulewidth}{0.2pt}

% i Sverige har vi normalt inget indrag vid nytt stycke
\setlength{\parindent}{0pt}
% men däremot lite mellanrum
\setlength{\parskip}{10pt}

\maketitle

\section{Bakgrund}
I den digitala tidsåldern har tillgänglighet på webben blivit en central fråga.
Internet är en grundläggande del av vårt dagliga liv, och det är nödvändigt att se till att webbplatser och
digitalt innehåll är tillgängliga för alla användare, oavsett eventuella funktionsnedsättningar.
Många länder och organisationer har därför infört webbtillgänglighetslagar och riktlinjer för att främja
inklusion och tillgänglighet på nätet.
Dessa riktlinjer kallas för WCAG(Web Content Accessibility Guidelines) och skapades av W3C (World Wide Web Consortium).
Riktlinjerna måste följas av statliga, kommunala eller offentligt styrda organ
(enheter som är styrda eller finansierade av offentliga myndigheter eller regeringar).
\subsection{Nivåer}
WCAG lagar har olika nivåer, Nivå A innehåller de mest grundläggande och lägsta kraven för webbtillgänglighet vilket gör det
enklare för personer med funktionsnedsättningar (detta inkluderar även syn och hörsel).
Nivå AA är en högre grad av tillgänglighet än nivå A, detta gör den mer tillgänglig för en mer utbredd publik än för nivå A.
Nivå AAA är den högsta nivån av webbtillgänglighet, denna nivå finns det inget krav på att någon hemsida skulle behöva uppfylla
eftersom de är ofta så oerhört tidskrävande eller svåra att uppfylla, nivå AAA är den högsta ambitionsnivån, slutmålet för tillgänglighet.
\subsection{Verktyg}
Hemsidorna/programmen jag kommer använda för att kolla webbtillgängligheten på Arbetsgivarverket.se är ''Wave''(Web Accessibility Evaluation Tools),
''Lighthouse'' och ''aXe''.
Lighthouse är ett gratis verktyg som utvecklats av Google som kan hjälpa dig att upptäcka och förbättra fel
som finns på din eller någon annans hemsida. Den kan även hjälpa till med att identifiera om en hemsida
följer Webbtillgänglighetslagarna. Medans aXe är en variant skapade av ''Deque''.
Preety Kumar är VD för Deque Systems och grundade Deque 1999 med visionen att förena webbåtkomst,
både ur användar och teknikperspektiv.


(https://www.tillganglighetskrav.fi/lagar-och-standarder/wcag-2-1/)


Min vetenskapliga undersökning syftar till att granska och utvärdera webbtillgängligheten
på en specifik hemsida (i detta fall Arbetsgivarverket.se) och att bedöma
i vilken utsträckning den följer de gällande webbtillgänglighetslagarna och riktlinjerna.
\section{Avgränsningar}
Sidan i detta fall är Arbetsgivarverket.se.
Det kommer göras en övergripande undersökning för att ta reda på hur mycket den stämmer överrens med WCAG (2.1) riktlinjer
och sedan bedöma tillgängligheten på sidan.
Det är viktigt att notera att undersökningen inte kommer innehålla en bedömning av sidans kvalitet eller innehåll, utan fokuserar
helt på aspekter relaterade till webbtillgänglighet/WCAG.






\printbibliography

\end{document}
